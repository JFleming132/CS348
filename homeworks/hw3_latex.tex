% Only edit between \begin{questions} and \end{questions} tags.

\documentclass[12pt]{exam}
\usepackage[utf8]{inputenc}

\usepackage[margin=1in]{geometry}
\usepackage{amsmath,amssymb}
\usepackage{multicol}
\usepackage{listings}
\usepackage{enumerate}
\usepackage{blindtext}
\usepackage{scrextend}
\usepackage{graphicx}
\usepackage{comment}

\def\ojoin{\setbox0=\hbox{$\bowtie$}%
  \rule[-.02ex]{.25em}{.4pt}\llap{\rule[\ht0]{.25em}{.4pt}}}
\def\leftouterjoin{\mathbin{\ojoin\mkern-5.8mu\bowtie}}
\def\rightouterjoin{\mathbin{\bowtie\mkern-5.8mu\ojoin}}
\def\fullouterjoin{\mathbin{\ojoin\mkern-5.8mu\bowtie\mkern-5.8mu\ojoin}}


\renewcommand{\choiceshook}{%
    \setlength{\leftmargin}{15pt}%
}
\title{CS 348 - Homework 3 }
\author{Relational Algebra}
\date{Spring 2024}

\begin{document}

\maketitle
\noindent
Write your answers for questions 1 to 9 in this latex file. Use a latex editor, such as overleaf to edit and compile your latex to a PDF file. Submit your latex file to Brightspace. Submit your PDF file to Gradescope. Question 1 includes all relational algebra operators and symbols that you can use in your answers.


\begin{questions}
\question 

\textbf{Answer:} \\
    %.........  Write your answer   ..... %	
    $\sigma_{trip.id <= 7450}(trip)$ \\
\vspace{50 mm}
	
\question 

\textbf{Answer:} \\
    %.........  Write your answer   ..... %	
    $\sigma_{trip.id <= 7450}(trip)$ \\
\vspace{50 mm}

\question 

\textbf{Answer:} \\
    %.........  Write your answer   ..... %	
    $\pi _{origin\_station, start\_d, duration, destination\_station} (trips \bowtie (\rho_{name\rightarrow origin\_station}(\rho_{id\rightarrow start\_station\_id}(station))) \bowtie (\rho_{name\rightarrow destination\_station}(\rho_{id\rightarrow end\_station\_id}(station))))$ \\
\vspace{50 mm}

\question 

\textbf{Answer:} \\
    %.........  Write your answer   ..... %	
    $\pi_{bike_id} ((\sigma_{station.name="Mountain View Caltrain Station"} (trip \bowtie_{trip.id=station.end\_station\_id} station)) \cap (\sigma_{station.name="Evelyn Park and Ride"} (trip \bowtie_{trip.id=station.end\_station\_id} station)))$ \\
\vspace{50 mm}

\question 

\textbf{Answer:}  \\
    %.........  Write your answer   ..... %	
    $\pi_{station.id, station.name, trip.id, trip.duration} (station \leftouterjoin_{station.id=trip.end_station_id} trip)$ \\
\vspace{50 mm}

\question 
\textbf{Answer:} \\
    %.........  Write your answer   ..... %	
    $\pi_{station.id, station.name} (station) - \pi_{station.id, station.name} (station \bowtie_{station.id=trip.start\_station\_id} trip)$ \\
\vspace{50 mm}

\question 

\textbf{Answer:} \\
    %.........  Write your answer   ..... %	
    $\pi_{s1.name, s1.id, s2.name, s2.id} (\rho_{s1} (station) \times \rho_{s2} (station)) - \pi_{s1.name, s1.id, s2.name, s2.id} ((\rho_{s1} (station) \times \rho_{s2} (station)) \bowtie_{s1.id=start\_station\_id \; and \; s2.id=end\_station\_id} trip)$\\
\vspace{50 mm}

\question 

\textbf{Answer:} \\
    %.........  Write your answer   ..... %	
    $\pi_{station\_status.station\_id, station.name, station\_status.time} (\sigma_{station\_status.docks\_available=station.dock\_count} (station\_status \bowtie_{station\_status.station\_id=station.id} station))$\\
\vspace{50 mm}

\question 
\textbf{Answer:} \\
    %.........  Write your answer   ..... %	
    $trips - \pi_{t1} (\rho_{t1} (trips) \bowtie_{t2.duration > t1.duration} \rho_{t2} (trips)))$\\
\vspace{50 mm}

\question 
\textbf{Answer:} \\
    %.........  Write your answer   ..... 
    b. The resulting number of tuples will always be n. Since the union operator does not include duplicates, and $\pi_{A} S$ is a subset of $\pi_{A} R$, there will always be exactly n tuples. \\
    \\
    c. Minimum: 0. The minimum case occurs when n = m and B = C. Set differences cannot be calculated on sets that have different attributes, so B must be the same attributes as C and must be associated with the same keys.\\
    Maximum: n. If there exists no element in S where the attribute associated with the key in S is the same as the attribute associated with the key in R, then nothing will be removed from R.\\
    \\
    d. Minimum: 0. The minimum case occurs when there is one element in S and one element in R, both with the same key. Since there is no S.A strictly greater than any given R.A, nothing will be joined. \\
    Maximum: n * $\sum_{\i=0}^{m-1}i$. The maximum case occurs when every key in R appears in S. As such, for each key in A, we will join every key in S greater than the key in A, resulting in the given sum. \\
    \\
    e. Minimum: 1. The minimum case occurs when R and S have one element, and so only one combination of keys exists. \\
    Maximum: $m^2$. The maximum case occurs when n = m.
\vspace{50 mm}

\end{questions}
\bigskip 
\noindent


\end{document}